\subsection*{Trabalho 2 de Técnicas de programação}

\subsection*{Índice}


\begin{DoxyItemize}
\item 1 \href{#1-objetivo}{\tt Objetivo}~\newline
 $<$t$>$1.\+1 \href{#11-descrição}{\tt Descrição}~\newline

\item 2 \href{#2-requisitos}{\tt Requisitos}~\newline
 $<$t$>$2.\+1 \href{#21-utilizado}{\tt Utilizado}~\newline
 $<$t$>$2.\+2 \href{#22-instalações}{\tt Instalações}~\newline

\item 3 \href{#3-build-compilation}{\tt Build Compilation}~\newline

\item 4 \href{#4-diagramas}{\tt Diagramas}~\newline
 $<$t$>$ 4.\+1 \href{#41-diagrama-de-sequência}{\tt Diagrama de sequência}
\item 5 \href{#5-doxygen}{\tt Doxygen}
\item 6 \href{#6-visão-computacional}{\tt Visão Computacional}
\end{DoxyItemize}

\subsubsection*{1 Objetivo}

Aprender e trabalhar com a linguagem C++, implementando e aplicando conceitos ensinados em sala, incluindo detecção facial, banco de dados e manipulação do Git\+Hub. 

\subsubsection*{1.\+1 Descrição}

\paragraph*{Sistema de controle de acesso}

Simular um sistema para ser utilizado no Laboratório de Informática (L\+I\+NF) da Universidade de Brasília que faz o controle de acesso, modelando por cadastramento, tipo de usuário, detecção facial e reservas. 

~\newline


\subsubsection*{2 Requisitos}


\begin{DoxyItemize}
\item Compilador g++
\item Editor de texto
\item Biblioteca de processamento de imagens
\item Ferramenta de versionamento
\item Framework para criação de interface gráfica
\item Banco de dados
\end{DoxyItemize}

\paragraph*{2.\+1 Utilizado}


\begin{DoxyItemize}
\item Terminal linux
\item Compilador g++ Ubuntu 6.\+2.\+0-\/5ubuntu12 20161005
\item Atom
\item Git\+Hub
\item Open\+CV
\item Qt Creator 5.\+9
\item S\+Q\+Lite
\end{DoxyItemize}

\paragraph*{2.\+2 Instalações}

\subparagraph*{Compilador g++}


\begin{DoxyCode}
1 $ sudo apt-get install g++
\end{DoxyCode}


\subparagraph*{Atom}


\begin{DoxyCode}
1 https://atom.io/
\end{DoxyCode}


\subparagraph*{Git\+Hub}


\begin{DoxyCode}
1 $ sudo apt-get install git
\end{DoxyCode}
 \#\#\#\#\# Open\+CV 
\begin{DoxyCode}
1   Tutorial criado por: Flavio Amaral e Silva; Lukas Lorenz de Andrade; Marcella Pantarotto.
2 
3   Para baixar a OPENCV-3.2.0 com a Contrib (extensão) use os seguintes comandos no terminal:
4   - Para baixar
5 
6     wget https://codeload.github.com/opencv/opencv\_contrib/tar.gz/3.2.0
7     wget https://codeload.github.com/opencv/opencv/tar.gz/3.2.0
8 
9   - Para mover para o diretório home
10 
11     mv Downloads/opencv\_contrib-3.2.0.tar.gz ./
12     mv Downloads/opencv-3.2.0.tar.gz ./
13 
14   - Para extrair as pastas
15 
16     tar xvfz opencv-3.2.0.tar.gz
17     tar xvfz opencv\_contrib
18 
19   - Agora, abra o repositório opencv-3.2.0 e edite o arquivo CMakeLists.txt da seguinte forma:
20 
21   Delete:
22 
23   404       if(NOT DEFINED OPENCV\_CONFIG\_INSTALL\_PATH)      
24   405         math(EXPR SIZEOF\_VOID\_P\_BITS "8 * $\{CMAKE\_SIZEOF\_VOID\_P\}")
25   406         if(LIB\_SUFFIX AND NOT SIZEOF\_VOID\_P\_BITS EQUAL LIB\_SUFFIX)
26   407           ocv\_update(OPENCV\_CONFIG\_INSTALL\_PATH lib$\{LIB\_SUFFIX\}/cmake/opencv)
27   408         else()
28   409           ocv\_update(OPENCV\_CONFIG\_INSTALL\_PATH share/OpenCV)
29   410         endif()
30   411       endif()
31 
32   E substituia por:
33 
34   403       ocv\_update(OPENCV\_CONFIG\_INSTALL\_PATH lib$\{LIB\_SUFFIX\}/cmake/opencv)
35 
36   OBS: Os números são apenas indicativos das linhas em que é necessária a operação.
37   Salve o arquivo e feche o txt.
38 
39   - Abra o repositório opencv\_contrib-3.2.0/modules/optflow/samples/ e edite o arquivo motempl.py da
       seguinte forma:
40   Substituia a linha 40:
41 
42   39        if ret == False:
43   40            break
44   41        h, w = frame.shape[:2]
45   42        prev\_frame = frame.copy()
46 
47   Por:
48   40            print("could not read from video source")
49   41            sys.exit(1)
50 
51   Salve o arquivo txt e feche-o.
52 
53   Execute os seguintes comandos:
54 
55     cd opencv-3.2.0
56     mkdir build
57     cd build
58 
59     cmake -D CMAKE\_BUILD\_TYPE=Release -D CMAKE\_INSTALL\_PREFIX=/usr/local -D
       OPENCV\_EXTRA\_MODULES\_PATH=../../opencv\_contrib-3.2.0/modules ..
60 
61     make -j4
62 
63     sudo make install
64 
65     sudo ldconfig
66 
67 Pronta para uso!! ( ͡° ͜ʖ ͡°)
\end{DoxyCode}
 \paragraph*{Qt Creator}

\href{https://wiki.qt.io/Install_Qt_5_on_Ubuntu}{\tt Passo-\/a-\/passo}

\paragraph*{S\+Q\+Lite}


\begin{DoxyCode}
1 $ sudo apt-get install sqlite3 libsqlite3-dev
\end{DoxyCode}


\subsubsection*{3 Build Compilation}

Entrar na pasta Projeto\+T\+P1 e usar o seguinte código 
\begin{DoxyCode}
1 $ cd ProjetoTP1
2 $ make ProjetoTP1.pro -spec linux-g++
3 $ make all
4 $ make clean
5 $ chomd 775 ProjetoTP1\_EXEC
6 ./ProjetoTP1\_EXEC
\end{DoxyCode}


\subsubsection*{4 Diagramas}

\paragraph*{4.\+1 Diagrama de Sequência}



\subsubsection*{5 Doxygen}

\mbox{[}Documentação Doxygen\mbox{]} Pasta H\+T\+ML e Latex

\subsubsection*{6 Visão Computacional}

Uso da biblioteca O\+P\+E\+N\+CV 3.\+2 e do módulo de reconhecimento facial parar a implementação do sistema de login via detecção facial
\begin{DoxyItemize}
\item Uso do Cascade lbpcascade\+\_\+frontalface.\+xml, para a deteção facil.
\item Model para trainamento e reconhecimento usando o algoritmo L\+B\+PH de reconhecimento \+:
\begin{DoxyItemize}
\item radius = 3, neighbors = 12, grid\+\_\+x = 12, grid\+\_\+y = 12, threshold = 320.\+0
\item treinamento do model com 10 fotos.
\end{DoxyItemize}
\end{DoxyItemize}

\subsubsection*{7 Prints}